\documentclass[11 pt,a4paper]{article}

%Chargement des paquets
\usepackage[utf8]{inputenc}
\usepackage{natbib}
\usepackage[english]{babel}   
\usepackage{graphicx}
\usepackage{lmodern}
\usepackage{amsmath}
\usepackage{tikz}
\usetikzlibrary{calc}
\usepackage{subfig}
%\usepackage{layout}
\usepackage{longtable}
\usepackage{epsfig}
\usepackage{color}
%\usepackage{esint} %Pour intégrale barrée...
\usepackage{setspace}
\usepackage[top=2cm,bottom=2.5cm,left=2.5cm,right=2.5cm]{geometry}
\usepackage{sectsty}
\usepackage{fancyhdr}
\usepackage{wasysym}
\usepackage[colorlinks=true]{hyperref}
\usepackage{caption}
\captionsetup{figurewithin=none}
\captionsetup{tablewithin=none}
\usetikzlibrary{patterns}
\usepackage{amsmath}
\usepackage{amsfonts}
\usepackage{amssymb}
\usepackage{amsthm}


%Définition des marges internes
\setlength{\headheight}{41 pt}
\setlength{\headsep}{20 pt}
\setlength{\footskip}{30 pt}

                 
%Definition des headers et footers
\pagestyle{fancy}
\renewcommand{\footrulewidth}{0.4 pt}
\renewcommand{\headrulewidth}{0.4 pt}
\fancyhead[L]{~}
\fancyhead[C]{~}
\fancyfoot[L]{~}
\fancyfoot[C]{\thepage}
\fancyfoot[R]{~}


%\renewcommand\contentsname{Table of contents}
\newtheorem{theorem}{Theorem} %[subsection]
\newtheorem{definition}[theorem]{Definition}
\newtheorem{lemme}[theorem]{Lemma}
\newtheorem{proposition}[theorem]{Proposition}

%Macros
\newcommand\TODO[1]{\textcolor{red}{#1}}
\newcommand{\tr}{^{\mathsf{T}}}
\newcommand{\Real}{\mathbb{R}}
\newcommand{\veps}{\varepsilon}


\title{Notes PDE Miura Ori}
\author{\begin{minipage}{\textwidth}\centering Fr\'ed\'eric
Marazzato$^{1}$\\
   \small{$^{1}$Department of Mathematics, Louisiana State University, Baton Rouge, LA 70803, USA}\\
   \small{email: \texttt{marazzato@lsu.edu}}\end{minipage}}
      
%\date{}
   
\begin{document}
%\hypersetup{urlcolor=blue,linkcolor=red,citecolor=blue}

\maketitle

\begin{abstract}
Writing the PDE to compute the Miura surface for given boundayr condition. Studying existence and uniqueness or solutions. Numerical method to compute solutions.
\end{abstract}

\section{Minimisation problem}

\subsection{System of PDEs}

$\varphi : \Omega \subset \mathbb{R}^2 \rightarrow \mathbb{R}^3$ is the position of the manifold we are trying to compute.

\begin{equation}
\label{eq:strong form equations}
\left\{
\begin{aligned}
& \frac{\varphi_{,xx}}{1 - \frac14 |\varphi_{,x}|^2} + \frac{4\varphi_{,yy}}{|\varphi_{,y}|^2} = 0 \in \mathbb{R}^3,\\
& \left\langle \varphi_{,x} ; \varphi_{,y} \right\rangle = 0, \\
& (1 - \frac14 |\varphi_{,x}|^2) |\varphi_{,y}|^2 = 1, \\
& (|\varphi_{,x}|^2, |\varphi_{,x}|^2) \in ]0,3]\times ]1,4]. \\
\end{aligned}
\right.
\end{equation}

We consider a test function $\psi \in H^1_0(\Omega)$, the first equation writes in a weak form:
\begin{equation}
\label{eq:weak form}
\int_{\Omega} 2\mathrm{Argtanh}(\frac12 |\varphi_{,x}|)(\psi_x + \psi_y) - \frac{4}{|\varphi_{,y}|}(\psi_x + \psi_y) = 0, \quad \forall \psi \in H^1_0(\Omega)
\end{equation}
The associated energy is then: \TODO{Modifier}
\begin{equation}
\label{eq:energy}
\int_{\Omega} 2\left( \int \mathrm{Argtanh} \right)(\frac12 |\varphi_{,x}|)(\psi_x + \psi_y) - \frac{4}{|\varphi_{,y}|}(\psi_x + \psi_y) = 0, \quad \forall \psi \in H^1_0(\Omega)
\end{equation}
A primitive of $\mathrm{Artanh}$ is \TODO{Write it after checking it...}

The equality constraints write:
\begin{equation}
\Lambda_1 \int_{\Omega} \left\langle \varphi_{,x} ; \varphi_{,y} \right\rangle + \Lambda_2 \int_{\Omega} \left( (1 - \frac14 |\varphi_{,x}|^2) |\varphi_{,y}|^2 - 1 \right),
\end{equation}
with $\Lambda_1$ and $\Lambda_2$ Lagrange multipliers.

The inequality constraints write:
%\begin{equation}
%\label{eq:inequality constraints energy}
%\lambda_1 \int_{\Omega} \left(\frac12 |\varphi_{,x}|^2 - \sqrt{3} |\varphi_{,x}| \right) + \lambda_2 \int_{\Omega} \left(\frac12 |\varphi_{,y}|^2 - 2 |\varphi_{,y}| \right) + \lambda_3 \int_{\Omega} \left(|\varphi_{,y}| - \frac12 |\varphi_{,y}|^2 \right)
%\end{equation}
\begin{equation}
\label{eq:inequality constraints energy}
\lambda_1 \int_{\Omega} \left(|\varphi_{,x}| - \sqrt{3} \right)^\oplus + \lambda_2 \int_{\Omega} \left(|\varphi_{,y}| - 2 \right)^\oplus + \lambda_3 \int_{\Omega} \left(1 - |\varphi_{,y}| \right)^\oplus + \lambda_4 \int_{\Omega} \left(\varepsilon - |\varphi_{,x}| \right)^\oplus,
\end{equation}
with $\lambda_1,\lambda_2,\lambda_3,\lambda_4$ penalties \TODO{or Lagrange mutlipliers?} and $\varepsilon = o(1)$.

\TODO{Or write a variational inequality due to that? Then we would have a minimisation of the Lagrangian with equality constraint under inequality constraint.}

\bibliographystyle{plain}
\bibliography{bib}


\end{document}
